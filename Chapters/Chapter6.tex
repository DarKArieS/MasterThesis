\chapter{Results} % Main chapter title

\label{Chapter6} % Change X to a consecutive number; for referencing this chapter elsewhere, use \ref{ChapterX}
%----------------------------------------------------------------------------------------
%	SECTION 1
%----------------------------------------------------------------------------------------
% \section{Limits}

% use asymptotic method in combine tool
% mention that combines the HM and LM 

%----------------------------------------------------------------------------------------
%	SECTION 2
%----------------------------------------------------------------------------------------

The procedure of limits extraction follows Ref.~\cite{1007.1727}, with Higgs Analysis Combination tools~\cite{CombinedTool}.
The high purity and medium purity categories are combined for all results.

\section{Resonance results}

The observed and expected upper limits at 95\% confidence level (CL) are shown in Fig.~\ref{fig:ResLimit}, for the cross-section of $pp \rightarrow X \rightarrow \HHbbgg$ process assuming spin-0 and spin-2 heavy resonances.
The observed data are from 0.23 fb to 4.2 fb, depending on the resonance mass hypotheses.

The results are compared with the cross sections for the spin-0 radion and spin-2 KK-graviton production in the bulk RS WED models.
In radion case, the cross-section is proportional to the parameter $1/ \Lambda^{2}_{R}$.
In the Run I results, $\Lambda_{R}$ = 1 TeV is excluded below 980 GeV in the radion search and have no sensitivity at $\Lambda_{R}$ = 3 TeV.
In this analysis, we compare the results of the radion search with  $\Lambda_{R}$ = 2 TeV and $\Lambda_{R}$ = 3 TeV.
The observed limits exclude the points of $m_{x}$ below 840 GeV in $\Lambda_{R}$ = 2 TeV and of $m_{x}$ below 540 GeV in $\Lambda_{R}$ = 3 TeV.
In the graviton case, the parameter for the properties of graviton $k/\overline{M}_{Pl}$ equals to 1.0 and 0.5 are compared.
With assuming $k/\overline{M}_{Pl} = 1.0$, the range $290<m_{x}<810$ GeV is excluded.
With assuming $k/\overline{M}_{Pl} = 0.5$, the range $350<m_{x}<530$ GeV is excluded.

\begin{figure*}[bth]
  \centering
  \includegraphics[width=0.45\textwidth]{Figures/Chapter6/ResRadLimit}\hfil
  \includegraphics[width=0.45\textwidth]{Figures/Chapter6/ResGraLimit}\hfil
  \caption{Observed and expected 95\% CL upper limits on the cross-section $\sigma(pp\rightarrow X \rightarrow HH \rightarrow b\bar{b}\gamma\gamma)$ combining two categories in high mass region and low mass region. The limits of 600GeV are set in both high mass and low mass methods. The green and yellow bands present one and two standard deviation, respectively. The theoretical predictions in WED with different parameters are compared.}
  \label{fig:ResLimit}
\end{figure*}


%----------------------------------------------------------------------------------------
%	SECTION 3
%----------------------------------------------------------------------------------------

\section{Non-resonance results}

The observed (expected) 95\% CL upper limits of SM-like di-Higgs production on the cross-section of $pp \rightarrow \HHbbgg$ are 2.0 (1.6) fb, which can be translated into 0.79 (0.63) pb for the cross-section of $pp \rightarrow HH$ corresponding to about 24 (19) times the SM prediction.

The results are also performed in the anomalous couplings study.
The results of limits on the BSM benchmark points in 5-D BSM coupling space are shown in Fig.~\ref{fig:BenchLimit}, which provide the constraint in different regions.
The tightest constraint happens on the benchmark 2, which has a wide peak around 800 GeV in $m_{HH}$ spectrum and has a lot of events in the high mass category.
The loosest constraint is placed on the benchmark 7, which has most events locating below $m_{HH}$ = 300 GeV and in low mass category.
It is observed that the high mass category has good sensitivity.

In addition, the scanning of $\kappa_{\lambda}$ with the other parameters keeping in the SM value ($\kappa_{t}$ = 1, other BSMs = 0) is also searched.
The 95\% CL limits are set on the cross-section as a function of $\kappa_{\lambda}$, and are shown in Fig.~\ref{fig:KlLimit}.
The $\kappa_{\lambda}$ is constrained in the range $-11 < \kappa_{\lambda} < 17$ by the observed data.

\begin{figure}[h]
  \centering
  \includegraphics[width=0.8\textwidth]{Figures/Chapter6/BenchLimit}
  \caption{Observed and expected 95\% CL upper limits on the cross-section $\sigma(pp\rightarrow X \rightarrow HH \rightarrow b\bar{b}\gamma\gamma)$ in recommended benchmark points.}
  \label{fig:BenchLimit}
\end{figure}

\begin{figure}[h]
  \centering
  \includegraphics[width=0.8\textwidth]{Figures/Chapter6/KlLimit}
  \caption{Observed and expected 95\% CL upper limits on the cross-section $\sigma(pp\rightarrow X \rightarrow HH \rightarrow b\bar{b}\gamma\gamma)$ as a function of $\kappa_{\lambda}$. Other parameters are the same as the SM.}
  \label{fig:KlLimit}
\end{figure}



% \section{Combination with other channel}


\chapter{Conclusion}
% \addcontentsline{toc}{chapter}{Conclusion}

This thesis is devoted for the search of the di-Higgs boson (HH) production decaying into the $b\bar{b}\gamma\gamma$ in the proton-proton collisions at $\sqrt{s}$ = 13TeV during the LHC Run II.
The data are collected by the CMS detector, which correspond to an integrated luminosity of 35.9 $fb^{-1}$.
Both the Standard Model and BSMs are searched for the resonant and the non-resonant HH productions.
The new developed techniques are updated in this analysis, and the new energy regression for the b jets improves about 10\% in the sensitivity.
Upper limits at a 95\% confidence level are set on the cross sections for the HH production decaying to two photons and two b jets.
No statistically significant deviations from the expectations are found in all searches.
The new particles in WED model with spin-0 (radion) and spin-2 (graviton) decaying into HH boson are searched in the mass range between 250 GeV and 900 GeV.
The observed limits are from 0.23 fb to 4.2 fb, which exclude the radion (spin-0) signal hypothesis, assuming the scale parameter $\Lambda_{R}$ = 3 TeV, for all masses below $m_{X}$ = 550 GeV, and the KK-graviton (spin-2) hypothesis for the mass range 280 < $m_{X}$ < 800 GeV, assuming $\kappa/\overline{M}_{Pl}$ = 1.
The observed limit on the cross-section of the SM-like HH production is 2.0 fb, which is about 24 times the SM prediction.
The anomalous couplings for the BSMs approach are also searched.
The deviation of the Higgs self-coupling constant in the SM is scanned and constrained in the range $-11 < \kappa_{\lambda} < 17$.
The recommended points in the anomalous couplings space with specific kinematics shape are also observed.
There is no significant data excess in all of points.
