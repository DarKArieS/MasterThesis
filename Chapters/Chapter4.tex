\chapter{Event selection} % Main chapter title

\label{Chapter4} % Change X to a consecutive number; for referencing this chapter elsewhere, use \ref{ChapterX}

%----------------------------------------------------------------------------------------
%	SECTION 1
%----------------------------------------------------------------------------------------
\section{Datasets}

\subsection{Data}
The data used in this analysis are collected by the CMS in full 2016, corresponding to an integrated luminosity of 35.9 /fb recorded by the CMS detector.
The ``DoubleEG'' dataset is used in this analysis, which is off-line reconstructed in 2017 and collects the events firing the specific di-electron/di-photon triggers. % ** need to be checked
% The di-photon HLT trigger \\ \verb|HLT_Diphoton30_18_R9Id_OR_IsoCaloId_AND_HE_R9Id_Mass90| 
% is used in this analysis as the same used in $\Hgg$ study.~\cite{1804.02716}
In this analysis, we use di-photon HLT trigger which is the same used in $\Hgg$ study.~\cite{1804.02716}
This HLT is seeded by the hardware L1 trigger electromagnetic candidates.
There should be a single candidate with transverse momentum higher than 25 GeV or a pair of candidates higher than 15 and 10 GeV respectively.
The HLT algorithm extends the L1 candidates (L1 seed) from the rectangle by $\Delta \eta \times \Delta \phi = 0.14 \times 0.4$ to be clusters.
There should exist the clusters passing the isolation plus calorimeter identification or $R_{9}$ selection, which is shown in Tab.~\ref{tab:HLT}.
Furthermore, the events should contain two clusters within $|\eta|<2.5$ having $E_{T}>30$ GeV and $E_{T}>18$ GeV, $H/E < 0.12$ and invariant mass of two objects above 90 GeV.
The rate of this HLT is 14 Hz during 2016 data taking.


\begin{table}[h]
\centering
\caption{The list of filters used in di-photon high level trigger (HLT).}
\label{tab:HLT}
\begin{tabular}{|c|l|l|l|l|}
\hline
\multicolumn{1}{|l|}{}                                                            & \multicolumn{1}{c|}{$R_{9}$(5x5)} & \multicolumn{1}{c|}{$\sigma_{i\eta i\eta}$(5x5)} & \multicolumn{1}{c|}{\begin{tabular}[c]{@{}c@{}}ECAL PF cluster\\ isolation\end{tabular}} & \multicolumn{1}{c|}{\begin{tabular}[c]{@{}c@{}}Track\\ isolation\end{tabular}} \\ \hline
\begin{tabular}[c]{@{}c@{}}Barrel region\\ $R_{9}$\textgreater{}0.85\end{tabular} & \textgreater 0.5                  & -                                                & -                                                                                        & -                                                                              \\ \hline
\begin{tabular}[c]{@{}c@{}}Barrel region\\ $R_{9}\leq$0.85\end{tabular}           & \textgreater 0.5                  & \textless 0.015                                  & \textless 6.0 + 0.012$E_{T}$                                                             & \textless 6.0 + 0.002$E_{T}$                                                   \\ \hline
\begin{tabular}[c]{@{}c@{}}Endcap region\\ $R_{9}$\textgreater{}0.90\end{tabular} & \textgreater 0.8                  & -                                                & -                                                                                        & -                                                                              \\ \hline
\begin{tabular}[c]{@{}c@{}}Endcap region\\ $R_{9}\leq$0.90\end{tabular}           & \textgreater 0.8                  & \textless 0.035                                  & \textless 6.0 + 0.012$E_{T}$                                                             & \textless 6.0 + 0.002$E_{T}$                                                   \\ \hline
\end{tabular}
\end{table}

\subsection{Resonant Monte Carlo signal samples} \label{sec:ResonantMCsamples}

The signal samples is simulated at leading order (LO) by \\{\textsc{MadGraph5}\_a\textsc{mc@nlo}\xspace} 2.3.2~\cite{1405.0301,1407.0281} interfaced with LHAPDF6~\cite{1412.7420}.
The next-to-leading-order (NLO) parton distribution function set \\PDF4LHC15\_NLO\_MC~\cite{1504.06469,1510.03865,1506.07443,1412.3989,1410.8849} is used.
The events are processed with {\textsc{Pythia 8.212}\xspace}~\cite{1410.3012} with the tune CUETP8M1~\cite{1512.00815} for showering, hadronization, underlying event and pile-up.
The the CMS detector is simulated by {\textsc{Geant 4}\xspace}~\cite{Agostinelli2003}.
The Warped Extra Dimension model in bulk RS scenario with narrow-width resonance is chosen as the benchmark model.
The new heavy particles both spin-0 Radion and spin-2 Graviton with the resonance masses from 250 GeV to 900 GeV, which are produced through gluon-gluon fusion and decay into two Higgs bosons with the mass $m_{H}=125~$GeV, are simulated.
The particles are assumed having narrow decay width which is much smaller than the detector resolution and negligible. % ** Rafael: width is 1 GeV, PAS and others: 1 MeV ??
The used resonance mass points are summarized in Tab.~\ref{tab:HbbggSamples}.

The cross-section for theoretical interpretation can be found at Ref.~\cite{WEDXSgithub}.
The configuration cards of {\textsc{MadGraph5}\_a\textsc{mc@nlo}\xspace} can be found at Ref.~\cite{MadCard}.

\begin{table}[h]
\caption{The resonant signal samples used in this analysis. The label ``M-X'' indicates the resonant mass points.}
\centering
\begin{tabular}{ll}
\hline
Samples                                            & Number of events \\ \hline
GluGluToRadionToHHTo2B2G\_M-250\_narrow\_13TeV       & 50000           \\
GluGluToRadionToHHTo2B2G\_M-260\_narrow\_13TeV       & 50000           \\
GluGluToRadionToHHTo2B2G\_M-270\_narrow\_13TeV       & 50000           \\
GluGluToRadionToHHTo2B2G\_M-280\_narrow\_13TeV       & 49600           \\
GluGluToRadionToHHTo2B2G\_M-300\_narrow\_13TeV       & 50000           \\
GluGluToRadionToHHTo2B2G\_M-320\_narrow\_13TeV       & 49998           \\
GluGluToRadionToHHTo2B2G\_M-340\_narrow\_13TeV       & 50000           \\
GluGluToRadionToHHTo2B2G\_M-350\_narrow\_13TeV       & 49999           \\
GluGluToRadionToHHTo2B2G\_M-400\_narrow\_13TeV       & 49996           \\
GluGluToRadionToHHTo2B2G\_M-450\_narrow\_13TeV       & 49998            \\
GluGluToRadionToHHTo2B2G\_M-500\_narrow\_13TeV       & 49997            \\
GluGluToRadionToHHTo2B2G\_M-550\_narrow\_13TeV       & 49998            \\
GluGluToRadionToHHTo2B2G\_M-600\_narrow\_13TeV       & 49998            \\
GluGluToRadionToHHTo2B2G\_M-650\_narrow\_13TeV       & 49999            \\
GluGluToRadionToHHTo2B2G\_M-700\_narrow\_13TeV       & 49198            \\
GluGluToRadionToHHTo2B2G\_M-750\_narrow\_13TeV       & 49997            \\
GluGluToRadionToHHTo2B2G\_M-800\_narrow\_13TeV       & 49792            \\
GluGluToRadionToHHTo2B2G\_M-900\_narrow\_13TeV       & 49994            \\
GluGluToBulkGravitonToHHTo2B2G\_M-250\_narrow\_13TeV & 49799           \\
GluGluToBulkGravitonToHHTo2B2G\_M-260\_narrow\_13TeV & 49998           \\
GluGluToBulkGravitonToHHTo2B2G\_M-270\_narrow\_13TeV & 48400           \\
GluGluToBulkGravitonToHHTo2B2G\_M-280\_narrow\_13TeV & 50000           \\
GluGluToBulkGravitonToHHTo2B2G\_M-300\_narrow\_13TeV & 49200           \\
GluGluToBulkGravitonToHHTo2B2G\_M-320\_narrow\_13TeV & 50000           \\
GluGluToBulkGravitonToHHTo2B2G\_M-340\_narrow\_13TeV & 49998           \\
GluGluToBulkGravitonToHHTo2B2G\_M-350\_narrow\_13TeV & 50000           \\
GluGluToBulkGravitonToHHTo2B2G\_M-400\_narrow\_13TeV & 49999           \\
GluGluToBulkGravitonToHHTo2B2G\_M-450\_narrow\_13TeV & 49999            \\
GluGluToBulkGravitonToHHTo2B2G\_M-500\_narrow\_13TeV & 49198            \\
GluGluToBulkGravitonToHHTo2B2G\_M-550\_narrow\_13TeV & 49995            \\
GluGluToBulkGravitonToHHTo2B2G\_M-600\_narrow\_13TeV & 49998            \\
GluGluToBulkGravitonToHHTo2B2G\_M-650\_narrow\_13TeV & 50000            \\
GluGluToBulkGravitonToHHTo2B2G\_M-700\_narrow\_13TeV & 49998            \\
GluGluToBulkGravitonToHHTo2B2G\_M-750\_narrow\_13TeV & 49999            \\
GluGluToBulkGravitonToHHTo2B2G\_M-800\_narrow\_13TeV & 49999            \\
GluGluToBulkGravitonToHHTo2B2G\_M-900\_narrow\_13TeV & 49993            \\ \hline
\end{tabular}
\label{tab:HbbggSamples}
\end{table}

\clearpage

\subsection{Non-resonant Monte Carlo signal samples} \label{sec:NonResonantMCsamples}

The MC generator set used for non-resonance analysis is the same as the resonance signal MC samples.
Both SM-like HH production and BSM HH production through gluon-gluon fusion are simulated based on the effective field theory (EFT) Lagrangian, which contains five parameters related to the Higgs boson coupling strength and is already described in Sec.~\ref{sec:nonResTheory}.
It is not feasible to generate the simulation samples with all possible combination of the five parameters.
The mentioned shape benchmark technique is used to produce MC samples.
The events which are simulated by different value of parameters and have similar shape of kinematics distribution are grouped into a cluster.
The benchmark points represent these clusters to reduce the complexity of the analysis.
The first version of recommended 13 benchmark points are used to generate our signal samples.
They are listed in Tab.~\ref{tab:NonResSamples}, contain the SM-like, box-diagram only and other BSM values.
The configuration cards of {\textsc{MadGraph5}\_a\textsc{mc@nlo}\xspace} can also be found at Ref.~\cite{MadCard}.

To explore the phase space beyond the simulated benchmark points, the event weighting technique is employed to approach other points.~\cite{1710.08261}
It starts from the assumption that the HH production is a 2 $\rightarrow$ 2 scattering.
The two Higgs bosons are produced back-to-back in the azimuthal direction and have the same momentum.
The azimuthal angle of the two Higgs boson is isotropic.
Consequently, the kinematics can just be determined by two variables: invariant mass of Higgs boson pair $m_{HH}$ and the absolute value of the cosine of the polar angle of one Higgs boson with respect to the beam axis $|cos \theta^{*}|$.
The phase space in these two variables is sliced optimally into 59 bins in $m_{hh}$ and 4 bins in $|cos \theta^{*}|$.

The event weights are derived from the differential cross-section of each bin, which is extended from the total cross-section described in Ch.~\ref{Chapter1}, Eq.~\ref{eq:EffectiveXS}.
It has the same form of Eq.~\ref{eq:EffectiveXS} with coefficients ${A}_{i}^{j}$ in different bin $j$.
The differential cross-section becomes
\begin{equation} \label{eq:DiffXS}
  \begin{aligned}
	R_{HH}^{j}=\frac{\sigma_{HH}}{\sigma^{SM}_{HH}} \frac{Frac^{j}}{Frac^{j}_{SM}}=Poly({A}_{i}^{j}).
  \end{aligned}
\end{equation}
The $Frac^{j}$ is the fraction of events of the simulated BSM samples in bin $j$, and $Frac^{j}_{SM}$ is for the simulated sample of SM-like HH production.
The coefficients ${A}^{j}$ are extracted by the likelihood fit and scanning the space of the five parameters.
This procedure is similar to the procedure to extract the total cross-section coefficients.
To obtain the events of the arbitrary point in the space of five parameters, the all simulated events are summed together.
The events in bin $j$ are applied the weight:
\begin{equation} \label{eq:ReWeight}
  \begin{aligned}
	W_{j}=R_{HH}^{j} \frac{\sigma^{SM}_{HH}}{\sigma_{HH}} \frac{Frac^{j}_{SM}}{C_{norm}} ,  % ** need to check from the source code!
  \end{aligned}
\end{equation}
where $C_{norm} = \sum\limits_{j}~R_{HH}^{j}~\frac{\sigma^{SM}_{HH}}{\sigma_{HH}}~{Frac}^{j}_{SM}$ is the factor to limit the errors from the fit and simulated samples. In the ideal case, $C_{norm}$ is equal to one.

Fig.~\ref{fig:ReweightValidate} shows the comparison in different kinematics variables between two SM-like sample, one is from MC simulation and the other one is from re-weighting.
The good agreement is observed.

\begin{table}[h]
\centering
\caption{The list of non-resonant signal samples and the values of five parameters in 13 clusters. The box-diagram only sample is also produced.}
\label{tab:NonResSamples}
\begin{tabular}{ccccccc}
\hline
Samples    & \begin{tabular}[c]{@{}c@{}}Number \\ of events\end{tabular} & $\kappa_{\lambda}$ & $\kappa_{t}$ & $c_{2}$ & $c_{g}$ & $c_{2g}$ \\ \hline
Node1 (SM) & 49998                                                       & 1.0                & 1.0          & 0.0     & 0.0     & 0.0      \\
Node2      & 49598                                                       & 7.5                & 2.5          & -0.5    & 0.0     & 0.0      \\
Node3      & 50000                                                       & 15.0               & 1.5          & -3.0    & -0.0816 & 0.3010   \\
Node4      & 49996                                                       & 5.0                & 2.25         & 3.0     & 0.0     & 0.0      \\
Node5      & 49999                                                       & 10.0               & 1.5          & -1.0    & -0.0956 & 0.1240   \\
Node6      & 49998                                                       & 1.0                & 0.5          & 4.0     & -1.0    & -0.3780  \\
Node7      & 49998                                                       & 2.4                & 1.25         & 2.0     & -0.2560 & -0.1480  \\
Node8      & 49998                                                       & 7.5                & 2.0          & 0.5     & 0.0     & 0.0      \\
Node9      & 49600                                                       & 10.0               & 2.25         & 2.0     & -0.2130 & -0.0893  \\
Node10     & 49799                                                       & 15.0               & 0.5          & 1.0     & -0.0743 & -0.0668  \\
Node11     & 49998                                                            & -15.0              & 2.0          & 6.0     & -0.1680 & -0.5180  \\
Node12     & 49996                                                       & 2.4                & 2.25         & 2.0     & -0.0616 & -0.1200  \\
Node13     & 49997                                                       & -15.0              & 1.25         & 6.0     & -0.0467 & -0.5150  \\ \hline
NodeBox    & 49999                                                       & 0.0                & 1.0          & 0.0     & 0.0     & 0.0      \\ \hline
\end{tabular}
\end{table}

\begin{figure}[h]
  \centering
  \includegraphics[width=0.95\textwidth]{Figures/Chapter4/ReweightValidate}
  \caption{The comparison between SM-like simulated and re-weighting samples in three kinematics variables: the four body invariant mass, the angle between two Higgs bosons and the four body transverse momentum.~\cite{1710.08261}}
  \label{fig:ReweightValidate}
\end{figure}

\subsection{Background simulation} \label{sec:BkgMC}

The background is contributed mainly by n$\gamma$ + jets events in the phase space of our analysis.
They mainly come from the non-resonance QCD production of two photons and two jets.
Meanwhile, the jets misidentified as photons with multi-jets and other photons can be a source.
It is challenging to simulate the QCD events accurately due to large effects in high order.~\cite{1706.08309}
In this analysis, these contributions are estimated from the data.

On the other hand, the contribution of single Higgs production in \\ non-resonance search is not negligible.
The SM-like single Higgs boson production with two additional jets, where the Higgs boson decays into two photons, through gluon-gluon fusion (ggH), vector boson fusion (VBF H) and associated production with $t\bar{t}$ ($t\bar{t}$H), $b\bar{b}$ ($b\bar{b}$H), and vector boson (VH) is considered as a source of background.
They are simulated by {\textsc{MadGraph5}\_a\textsc{mc@nlo}\xspace} 2.2.2 for VH, 2.3.3 for $b\bar{b}$H, and {\textsc{powheg} 2.0}~\cite{hep-ph/0409146,0709.2092,1002.2581,1111.2854} at NLO for ggH, VBF H and $t\bar{t}$H.
They are also interfaced with {\textsc{Pythia 8.212}\xspace} and {\textsc{Geant 4}\xspace} as the signal samples.

%----------------------------------------------------------------------------------------
%	SECTION 2
%----------------------------------------------------------------------------------------
\section{Physical objects}

All of the final-state physics objects are reconstructed by the particle flow algorithm.~\cite{1706.04965}
This algorithm links all of the reconstructed ingredients together from the sub-detectors in the CMS and identify the particle type.
Each reconstructed object is mutually exclusion, where the ingredients are only used once for one objects.
The identification is benefited by the structure of the CMS detector, as shown in Fig.~\ref{fig:CMSPFsket}.
Starting from the interaction region, the charged particles is detected by the tracker first.
The electromagnetic particles are absorbed by the ECAL, and the hadronic particles are obstructed by the HCAL.
Finally, the escaping muons cause the hit in the muon system.
The particles can be identified according to the linked hits in each sub-detector.
The correlations between each sub-detector are taken into account to identify the properties of particles.


\begin{figure}[h]
  \centering
  \includegraphics[width=0.8\textwidth]{Figures/Chapter4/CMS-PRF-14-001_Figure_001}
  \caption{A sketch of the specific particle interactions in a transverse slice of the CMS detector.~\cite{1706.04965}}
  \label{fig:CMSPFsket}
\end{figure}

\subsection{The $\Hgg$ candidate}

\subsubsection{Trigger mimic pre-selection}

To achieve a good agreement between data and simulation, a pre-selection which is tighter than the trigger selection is applied in order to mimic the trigger.
This pre-selection is developed for the \Hgg search~\cite{1804.02716} and is employed in \HHbbgg analysis.
The scale factors and uncertainties related to this selection are also applied in this analysis.
The selection is summarized in Tab.~\ref{tab:HLTmimic}.

\begin{table}[h]
\centering
\caption{The trigger mimic selection for the photon trigger.}
\label{tab:HLTmimic}
\begin{tabular}{|c|c|c|}
\hline
\multirow{2}{*}{\begin{tabular}[c]{@{}c@{}}Barrel region\\ $|\eta|<1.4442$\end{tabular}}    & $R_{9}$(5x5)\textgreater{}0.85 & -                                                                                                                                              \\ \cline{2-3} 
                                                                                            & $R_{9}$(5x5)$\leq$0.85         & \begin{tabular}[c]{@{}c@{}}$R_{9}$(5x5)\textgreater{}0.5\\ pfPhoIso \textless 4 GeV\\ $\sigma_{i\eta i\eta}$(5x5)\textless{}0.015\end{tabular} \\ \hline
\multirow{2}{*}{\begin{tabular}[c]{@{}c@{}}Endcap region\\ $1.556<|\eta|<2.5$\end{tabular}} & $R_{9}$(5x5)\textgreater{}0.9  & -                                                                                                                                              \\ \cline{2-3} 
                                                                                            & $R_{9}$(5x5)$\leq$ 0.9         & \begin{tabular}[c]{@{}c@{}}$R_{9}$(5x5)\textgreater{}0.8\\ pfPhoIso \textless 4 GeV\\ $\sigma_{i\eta i\eta}$(5x5)\textless{}0.035\end{tabular} \\ \hline
\multicolumn{3}{|l|}{\begin{tabular}[c]{@{}l@{}}Pass electron-veto to reject the electron;\\ $R_{9}$(5x5) \textgreater 0.8 or chargeIso \textless{}20 GeV or chargeIso/$E_{T}$ \textless 0.3;\\ H/E \textless 0.08;\end{tabular}}                                             \\ \hline
\end{tabular}
\end{table}

\subsubsection{$\Hgg$ selection}\label{ssec:photonselection}
% need to maintion control region for categorization MVA training

The photon pair selection is based on the kinematics variables and the photon MVA identification which is described in Sec~\ref{sec:phoID}.
The selection is as follows:
\begin{itemize}
\item Leading photon divided by the di-photon invariant mass $p^{\gamma 1}_{T}/M_{\gamma\gamma}>1/3$, trailing photon $p^{\gamma 2}_{T}/M_{\gamma\gamma}>1/4$;
\item $100 < M_{\gamma\gamma} < 180$ GeV;
\item Both photons pass photon MVA identification with 90 \% efficiency working point;
\end{itemize}
The trigger efficiency of the events passing the selection above is 100 \%.

In addition, the control region is obtained by selecting the photon pair with only one photon passing the photon MVA identification.
The other procedures of selection are the same as the normal selection.
This control region is used by the training of categorization MVA, which is described in Sec.~\ref{sec:catMVA}, and also the background modeling validation, which is mentioned in Sec.~\ref{sec:biasstudy}.

\subsection{The $\Hbb$ candidate}

The jets in LHC Run-II at CMS are reconstructed by clustering the particle-flow objects with the anti-$k_{t}$ algorithm using cone size $\Delta R = 0.4$.~\cite{1126-6708-2008-04-063}
The cone size is smaller than LHC Run I analysis, which is $\Delta R = 0.5$, due to the higher luminosity and number of pile-up.
This change leads jets to be worse resolution, because there is less energy clustered.
The energy of jets are corrected by the flavor blind standard regression.~\cite{1107.4277} % ** this reference is for Run I, so old :(
In addition, the specific energy regression for b-jet is applied, which is described in Sec.~\ref{sec:breg}.
The jets are selected by the criteria as follows:
\begin{itemize}
\item Pass loose particle flow jet identification~\cite{CMS-PAS-JME-16-003};
\item $\pT>25$ GeV;
\item $|\eta|<2.4$, which is within the tracker of the CMS and can be tagged as b-jet;
\item Outside the cone of selected photons with $ \Delta R(jet,\gamma) > 0.4$;
\item $70 < M_{jj} <90$;
\item The jet pair which has the highest sum of b-tag score is selected.
\end{itemize}


\subsection{The di-Higgs system} \label{ssec:diHiggsSelection}
% decribe Mx and mass window selection

After the di-photon and di-jet selection, the selected objects are combined to become a di-Higgs boson candidate.
To minimize the dependence on the resolution of photons and jets, the effective four body invariant mass~\cite{1404.0996} is defined as:
\begin{equation} \label{eq:MxDef}
  \begin{aligned}
	\Mx=M_{jj\gamma\gamma}-M_{jj}-M{\gamma\gamma}+250~GeV.
  \end{aligned}
\end{equation}

This variable scales the di-photon and di-jet invariant mass to 125 GeV.
Fig.~\ref{fig:MxDistr} shows the comparison between the distributions of the four body invariant mass and the effective mass \Mx.
The impact on the low mass resonance is very huge.
% For the high mass resonance, the improvement is little since the resolution is much better originally.

\begin{figure}[h]
  \centering
  \includegraphics[width=0.8\textwidth]{Figures/Chapter4/MxDistr_new}
  \caption{Comparison between \Mx (red) and $m_{jj\gamma\gamma}$ (purple) distribution of spin-2 resonance with different masses.
  All of events pass the di-photon and di-jet selections.}
  \label{fig:MxDistr}
\end{figure}

The \Mx is used to restrict the signal phase space for the signal modeling in the resonance analysis.
The size of effective mass window around \Mx peak is optimized to reach the best sensitivity for each of searched resonance point.
The mass window is chosen to cover 60 \% signal shape in \Mx with the minimum range.
The exactly mass window range is a function of the resonance mass, which is fit by the polynomial and shown in Fig.~\ref{fig:MxWindows}.

\begin{figure}[h]
  \centering
  \includegraphics[width=0.6\textwidth]{Figures/Chapter4/MxWindows}
  \caption{The upper bounds and lower bounds of the window in \Mx as a function of searched resonance mass.
  The values of bound are fit by a 3rd polynomial.}
  \label{fig:MxWindows}
\end{figure}

\subsection{Signal selection efficiency}

The signal step-by-step selection acceptance times efficiency as a function of resonance mass is shown in Fig.~\ref{fig:effPlot}, which includes the online (trigger mimic) selection, the di-photon selection, the di-jet selection and the MVA categorization in different regions.
The final efficiencies range from approximately 20\% (low mass) to 50\% (high mass) for both spin-0 and spin-2 hypotheses.
In the non-resonance case, the efficiency is 30\% for the SM-like HH production, with 25\% in the high mass region and 5\% in the low mass region.


\begin{figure}[h]
  \centering
  \includegraphics[width=0.9\textwidth]{Figures/Chapter4/effPlot}
  \caption{Consecutive selection efficiencies step-by-step for two resonance hypotheses: spin-0 (left) and spin-2 (right).}
  \label{fig:effPlot}
\end{figure}
