\chapter{The LHC Machine and the CMS detector} % Main chapter title

\label{Chapter2} % Change X to a consecutive number; for referencing this chapter elsewhere, use \ref{ChapterX}

%----------------------------------------------------------------------------------------
%	SECTION 1
%----------------------------------------------------------------------------------------
This thesis is based on the data of proton-proton collisions recorded by the Compact Muon Solenoid (CMS) at the Large Hadron Collider at CERN.
The protons are accelerated at the energy of 6.5 TeV by the LHC and collided.
This chapter presents the overview of the LHC and the CMS.

\section{The Large Hadron Collider}

The Large Hadron Collider is the largest and most powerful particle accelerator in the world.
It consists of a 26.7 km superconducting ring with a number of accelerating structures, underground about 100 meters. 
Fig.~\ref{fig:LHC} shows the location and the accelerator complex.
The LHC is designed to accelerate and collide proton beams in the centre-of-mass energy of protons at 14 TeV with luminosity of $10^{34} cm^{-2}s^{-1}$.~\cite{Evans:2008zzb}

The protons are born from the hydrogen atoms which are ionized by an electric field.
The journey of protons begins in the linear accelerator, Linac2. The protons are accelerated to 50 MeV.
Then the protons beam is injected into the Proton Synchrotron Booster (PSB), accelerated to 1.4 GeV, and fed to the Proton Synchrotron to be accelerated to 25 GeV.
Next, they are sent to the Super Proton Synchrotron to be accelerate to 450 GeV. 
Finally, they are transferred to the two separate beam pipes of the Large Hadron Collider (LHC), split up into clockwise and anti-clockwise direction.
The protons are accelerated to 6.5 TeV in the LHC, travel in the speed about 99.99\% times the speed of light.
The superconducting dipole magnets providing about 8 $T$ magnetic field are used to keep the particles in their circular orbits.
The collisions occur at the intersection points in the LHC rings. There are four intersection points used for the experiments: CMS, ATLAS, LHCb and ALICE.
The quadrupole magnets focus the two beams to perform a bunch crossing, enhance the probability of the collisions.
Under normal operating conditions, the LHC provides a proton beam containing 2808 bunches.
The bunch length is about 7.5 $cm$ and the spacing is 25 $ns$ (about 7.5 $m$). Each bunch is expected to contains about $1.15 \times 10^{11}$ protons. % ***

In 2016, the centre-of-mass energy of protons reached 13 TeV.
The maximum number of injected bunches are 2208.
The beam intensity is up to $2.5 \times 10^{14}$, about $1.25 \times 10^{11}$ protons in each bunch.
The peak instantaneous luminosity is $1.4 \times 10^{34} cm^{-2}s^{-1}$, 40\% above the design value.~\cite{LHCStatistics}
The delivered integrated luminosity is 40.82 $fb^{-1}$ and 37.76 $fb^{-1}$ are recorded by the CMS. (Fig.~\ref{fig:lumi})~\cite{CMSLumiPublic}

\begin{figure}[t]
  \centering
  \includegraphics[width=0.7\textwidth]{Figures/Chapter2/LHC.png}
  \includegraphics[width=0.7\textwidth]{Figures/Chapter2/LHC_complex.jpg}
  \caption{(Top) The LHC is located beneath the border between Switzerland and France. (Bottom) The accelerator complex of the LHC.}
  \label{fig:LHC}
\end{figure}

\begin{figure}[t]
  \centering
  \includegraphics[width=0.7\textwidth]{Figures/Chapter2/int_lumi_per_day_cumulative_pp_2016.pdf}
  \caption{The integrated luminosity delivered by the LHC and recorded by the CMS in 2016.}
  \label{fig:lumi}
\end{figure}

%----------------------------------------------------------------------------------------
%	SECTION 2
%----------------------------------------------------------------------------------------

\section{The Compact Muon Solenoid}

The Compact Muon Solenoid (CMS) is a general-purpose detector at the LHC ~\cite{Chatrchyan:2008aa},
designed to record the proton-proton collisions with high luminosity.
% At the designed luminosity, about 20 inelastic collisions will be superimposed and about 1000 particles are produced every 25 $ns$.
To increase the luminosity, the high intensity of protons released by the LHC can generate more inelastic interaction and produce about 1000 particles every 25 $ns$.
In 2016, the average number of interaction vertices produced per bunch crossing, called pile-up, is up to 27 as shown in Fig~\ref{fig:CMS_PU}, which is about 25\% higher than 2012.
The high pile-up effect is a grand challenge for reconstruction.
Therefore, the detectors have to feature high granularity and fast response.
The prime goals of the CMS are to verify the Standard Model and also to explore other BSM (beyond Standard Model) physics at TeV energy scale.
The CMS detector is cylindrical symmetry around the beam axis, composed of a barrel and two endcaps with a diameter 15 $m$ and a length 28.7 $m$.
It is quite compact with a weight about 14000 tonnes.
The main components of the CMS are a superconducting solenoid electromagnet providing 3.8 Tesla magnetic field inside the solenoid,
silicon tracking system, electromagnetic calorimeter, hadronic calorimeter, and also a muon detecting system outside of the solenoid. 
The geometry of sub-detectors can be found in Fig.~\ref{fig:CMS}.

\begin{figure}[t]
  \centering
  \includegraphics[width=0.85\textwidth]{Figures/Chapter2/pileup_pp_2016.pdf}
  \caption{The distribution of number of pile-up per bunch crossing in the CMS in 2016. The maximum number is up to 53 and the average is 27.}
  \label{fig:CMS_PU}
\end{figure}

\begin{figure}[t]
  \centering
  \includegraphics[width=0.85\textwidth]{Figures/Chapter2/cms_120918_03.png}
  \caption{A perspective view of the CMS detector.}
  \label{fig:CMS}
\end{figure}

%-----------------------------------
%	SUBSECTION 
%-----------------------------------
\subsection{Magnetic system}

The magnetic system are a superconducting cylindrical coil with a length of 12.5 $m$ and a diameter of 6 $m$, designed to reach a uniform 4 $T$ field in a free bore.
The superconducting coils are Rutherford-type cables made by NbTi conductor, form as the 4-layer winding, total 2168 turns operating current 19.5 $kA$, storing energy 2.69 $GJ$.~\cite{CMS:MagPrjt}
The coils are in the cold box and cooled by the helium to 4.45 $K$ and insulated by a vacuum vessel.
There is a steel yoke outside the solenoid, where the muon system is housed, to return the magnetic flux. The yoke comprises 5 wheels and 2 endcaps, composed of three layers each. 
The magnetic field in the barrel and outer endcap return yoke is about 1.7 $T$ and points in a direction opposite to the direction of the field inside the coil.
Inside the solenoid, the tracking system, electromagnetic calorimeter, and the hadronic calorimeter are installed.
The muon system is outside the solenoid and interspersed in the return yoke.
This system used to bend charged particles flying outward from the collision point helps to identify the charge and improves the measurement of momentum.

\subsection{Tracking system} 

The inner tracking system (tracker) surrounds the collision points and occupies a cylindrical volume with a length of 5.8 $m$ and a diameter of 2.5 $m$. 
The solenoid provides a homogeneous magnetic field of 4 $T$ over the full tracker,
and is expected to reconstruct the trajectories of charged particles with transverse momentum \pT around 0.8 GeV. %% should up to 0.6 GeV
The reconstruction of trajectories is also used to reconstruct the primary vertex as well as secondary vertex, which is the most important part of the b-tagging technique.
The tracker is based on the silicon detector technology. The charged particles cause small ionization current which can be measured when they pass through the silicon.
It is composed of a pixel detector with 3 barrel layers and a silicon strip tracker with 10 barrel layers.
For the two endcaps, each comprises 2 disks of pixel detector and 3 inner disk plus 9 outer disks of the strip trackers.
The acceptance of the tracker is up to a pseudorapidity of $\vert\eta\vert < 2.5$.
The full layout is shown in Fig.~\ref{fig:CMSTracker} and described in Tab.~\ref{tab:TrackerLayout}.~\cite{1748-0221-9-10-P10009}
The size of each pixel cell is $100 \times 150$ $\mu m^{2}$, and provides the hit position resolution about 10 $\mu m$ in the transverse coordinate and 20-40 $\mu m$ in the longitudinal coordinate.
Total 1440 pixel modules are used with 66 million readout channels.
On the other hand, the strips tracker has 15148 modules with total 9.3 million strips, and provides position resolution in $r\phi$ approximately 13-38 $\mu m$ (TIB, TID), 18-47 $\mu m$ (TOB, TEC).
More details can be found in Ref.~\cite{Karimaki:368412}.

\begin{figure}[t]
  \centering
  \includegraphics[width=0.92\textwidth]{Figures/Chapter2/tracker.png}
  \caption{The layout of the trackers in r-z plane. The start point is the interaction point. The other components are Pixel Detector (PIXEL), Tracker Inner Barrel (TIB), Tracker Outer Barrel (TOB), Tracker Inner Disk (TID), and Track endcap (TEC).}
  \label{fig:CMSTracker}
\end{figure}


\begin{table}[t]
\caption{A summary table of the components of the tracker. Pitch in the barrel is the distancce between neighbouring strips. The barrel strips run parallel to the beam axis.}
\centering
\small
\begin{tabular*}{\textwidth}{llll}
\hline
Tracker subsystem&Layers&Pitch&Location\\
\hline
Pixel tracker barrel&3 cylindrical& $100 \times 150$ $\mu m^{2} $ & $4.4<r<10.2$ $cm$\\
Strip tracker inner barrel (TIB)&4 cylindrical& $80-120$ $\mu m $ & $20 < r < 55$ $cm$\\
Strip tracker outer barrel (TOB)&6 cylindrical& $122-183$ $\mu m $ & $55 < r < 116$ $cm$\\
\hline
Pixel tracker endcap&2 disks& $100 \times 150$ $\mu m^{2} $ & $34.5 < \vert z \vert < 46.5$ $cm$\\
Strip tracker inner disks (TID)&3 disks& $100-141$ $\mu m $ & $58 < \vert z \vert < 124$ $cm$\\
Strip tracker endcap (TEC)&9 disks& $97-184$ $\mu m $ & $124 < \vert z \vert < 282$ $cm$\\
\hline
\end{tabular*}
\label{tab:TrackerLayout}
\end{table}

\subsection{Electromagnetic calorimeter (ECAL)}

The electromagnetic calorimeter (ECAL) in CMS is a hermetic homogeneous calorimeter, is able to obstruct the electromagnetic (EM) particles and measures their energy deposits and positions.
One of the important tasks of ECAL is to explore the $H\rightarrow\gamma\gamma$ decay channel.
The lead tungstate crystals $PbWO_{4}$ are used to be scintillators, which have short radiation length ($X_{0}$ = 0.89 cm) to stop the EM particles, small Moliere radius (2.2 cm) to have small electromagnetic showers.
When EM particles strike a scintillator material, the material is excited and then emits light.
The scintillation emission spectrum of the ECAL crystals peaks at around 440 $nm$. The emitted light is received by the photo diodes to measure the energy deposits of the incoming EM particles.
About 80\% scintillation light is emitted in 25 $ns$. The time-response is fast to cope with the bunch crossing time of the LHC.
Although the crystal is radiation hard, the radiation damage still affectes the transparency of the crystals. The correction for the transparency loss is needed. ~\cite{1742-6596-587-1-012001}

ECAL is constructed with a barrel and two endcaps.
The ECAL barrel (EB) covers the pseudorapidity range $\vert\eta\vert < 1.479$.
It is constructed by 61200 crystals, and each of them is 22 $\times$ 22 $mm^{2}$ at the front face and 26 $\times$ 26 $mm^{2}$ at the rear face (0.0174 $\times$ 0.0174 in $\phi$-$\eta$), with a length 230 $mm$ (25.8 $X_{0}$). 
The ECAL endcap (EE) covers the pseudorapidity range $1.479 < \vert\eta\vert < 3$.
The crystal size is 28.62 $\times$ 28.62 $mm^{2}$ at the front face and 30 $\times$ 30 $mm^{2}$ at the rear face with a length 220 $mm$ (24.7 $X_{0}$).
To suppress the background $\pi^{0}\rightarrow\gamma\gamma$ for $H\rightarrow\gamma\gamma$, the spatial resolution should be improved. 
A sampling detector, Preshower (ES), installed in front of EE, covers the pseudorapidity range $1.65 < \vert\eta\vert < 2.61$.
Each endcap contains two lead absorbers and two planes of silicon strips which are orthogonal to each other.
The plane of silicon strips are behind the absorber to measure the EM showers from the incoming particles striking the absorber. The radiation length of the absorber is 2 $X_{0}$ for the front plane and 1 $X_{0}$ for the rear plane.
The whole layout is shown in Fig.~\ref{fig:CMSECAL} and Fig.~\ref{fig:CMSPreshower}.
More details can be found in Ref.~\cite{CMS:ECALprjt}.

\begin{figure}[t]
  \centering
  \includegraphics[width=0.92\textwidth]{Figures/Chapter2/ECAL.png}
  \caption{The layout of ECAL.}
  \label{fig:CMSECAL}
\end{figure}

\begin{figure}[t]
  \centering
  \includegraphics[width=0.92\textwidth]{Figures/Chapter2/Preshower.jpg}
  \caption{The layout of Preshower.}
  \label{fig:CMSPreshower}
\end{figure}

\clearpage

\subsection{Hadronic calorimeter (HCAL)}

The Hadron calorimeter is a sampling calorimeter surrounding the ECAL, designed to detect the strongly interacting particles and help to determine the missing energy.
The HCAL is composed of four parts: barrel (HB), endcap (HE), outer (HO) and forward (HF). The longitudinal scheme is shown in Fig.~\ref{fig:CMSHCAL}.
The barrel covers pseudorapidity range $\vert\eta\vert < 1.3$, between the outer extent of the ECAL (1.77 m) and inner extent of the magnetic coil (2.95 m). 
The absorbers in HCAL barrel consist a front steel plate with a thickness 40 $mm$, 8 layers brass plates with a thickness 50.5 $mm$, 6 layers brass plates with thickness 56.5 $mm$ and a back steel plate with a thickness 75 $mm$. Total thickness is 5.82 interaction lengths at $90^{\circ}$.
The brass materials are non-ferromagnetic and high nuclear interaction length, which are easy to be simulated and make the detector be more compact.
The plastic scintillator behind the absorber is segmented in $\eta-\phi$ towers granularity 0.087 $\times$ 0.087, which is equivalent to the area of the 5 $\times$ 5 ECAL crystals.
70000 tiles in the scintillators are connected to the wavelength shifting fibres which collect the emitted light and shift the blue-violet light to the green light.
The endcaps cover the region $1.3 < \vert\eta\vert < 3$. The absorbers are 79-mm-thick brass plates.
The granularity in $\eta-\phi$ is 0.087 $\times$ 0.087 for $\vert\eta\vert < 1.6$ and 0.17 $\times$ 0.17 for $\vert\eta\vert \geq 1.6$.
The outer calorimeter is located outside the magnetic coil, is designed to complete the containment for hadron showers of HB and HE.
The HO is placed as the first layer of the return yoke.
The magnetic coil is used as an additional absorber equal to $1.4/sin\theta$ interaction length.
At $\eta = 0$, the interaction length of the HB is minimum, there are two layers of scintillators with 19.5 cm thick piece of iron.
Other wheels have a single layer. The granularity of HO is same as HB.
Lastly, the two forward calorimeters (HF) are positioned at $3 < \vert\eta\vert < 5.2$, outside the muon chambers.
The HF will experience extremely high particles fluxes, about 10 MGy in the LHC with an integrated luminosity $5 \times 10^{5}$ $pb^{-1}$.
This calorimeter is sensitive to the EM component of showers, because the shower particles can achieve the energy above the Cherenkov threshold.
The absorber in the HF is a cylindrical steel with an inner radius 12.5 $cm$, an outer radius 130 $cm$ and a length 165 $cm$ along z-axis.
The granularity of HF is 0.175 $\times$ 0.175 in $\eta-\phi$ towers.
More details can be found in Ref.~\cite{CMS:HCALTDR}.

\begin{figure}[t]
  \centering
  \includegraphics[width=0.92\textwidth]{Figures/Chapter2/HCAL.png}
  \caption{Longitudinal view of the HCAL, showing the location of the barrel (HB), endcap (HE), outer (HO) and forward (HF) of HCAL.}
  \label{fig:CMSHCAL}
\end{figure}

\subsection{Muon system} \label{ssec:muon}

The particles produced from the collision are absorbed by the inner detectors with high probability.
The most particles escape outside are muons, which provide an unmistakable signature.
The muon system is able to identify muons, measure their momentum, and deal with the muon triggering.
It is composed of a barrel and two endcaps with three kinds of gaseous detectors. 
In the barrel region at $\vert\eta\vert < 1.2$, the muon rate is low and the 4-T magnetic field is quite uniform in the return yoke.
The drift tube (DT) chambers with 400 $ns$ drifting time are used, which is interspersed among the layers of the return yoke, total 4 stations.
3 stations are consisted of 8 chambers to measure in $r-\phi$ bending plane, and 4 chambers measure in $z$ direction.
The other one station doesn't include the $z$-direction measurement, but provides the best angular resolution.
The drift cells of each chamber are shifted by a half-cell to eliminate the gaps.
In both endcaps at $0.9 < \vert\eta\vert < 2.4$, the muon rates are high and the magnetic field is non-uniform.
The Cathode strip Chambers (CSC) are used for the endcaps, which have fast response and high radiation resistance.
There are 4 stations in each endcap interspersed in the return yoke. Each strip is run radially outward and provides $r-\phi$ plane measurement.
The anode wires are also used to measure the $\eta$ position and the beam-crossing time. They are perpendicular to the cathode strips.
To have a better muon triggering, the resistive plate chambers (RPC) are added in both the barrel and endcaps at $\vert\eta\vert < 1.6$.
The RPCs provide a fast response with good time resolution but coarse spatial resolution.
It also helps to resolve the tracks from the multi-hits in a chamber.
Full layout is shown in Fig.~\ref{fig:CMSMuon}.
More details can be found in Ref.~\cite{CMS:MuonTDR}.

\begin{figure}[t]
  \centering
  \includegraphics[width=0.92\textwidth]{Figures/Chapter2/CMSMuon.png}
  \caption{Layout of one quadrant of the muon system.}
  \label{fig:CMSMuon}
\end{figure}

\subsection{Trigger and data acquisition system}

At the designed luminosity in the LHC, there are about $10^{9}$ events generated per second, 
which is impossible to process and store every event. Therefore, it is important to reduce the data rate and select the events we interest.
The trigger system is the start of the event selection process in two steps -- level-1 trigger (L1) and high level trigger (HLT).

The level-1 trigger is composed by custom-designed, large programmable electronics based on FPGA with maximum output rate 100 kHz, which are installed partly on the detectors. (Other parts are in the control room.)
The L1 trigger has to analyze each bunch crossing every 25 $ns$ without any dead-time, but the trigger latency is 3.2 $\mu$$s$.
The data are stored in the pipelined memories first and wait for the L1 trigger decision transmitted to the detector electronics within 3.2 $\mu$$s$.
Due to the 3.2 $\mu$$s$ restriction, the event selection by L1 should rely on the coarser data and simple algorithm, from the calorimeters and muon system.

There are three components of the L1: local, regional and global as shown in Fig.~\ref{fig:CMSTrigger}.
The local trigger is based on the calorimeters and the muon trigger, respectively.	
The calorimeter local triggers are done by the Trigger Primitive Generator circuit in each calorimeter cell and begins with the sum of the tower energy from ECAL, and HCAL.
The regional calorimeter trigger (RCT) combines the information of all the local triggers of ECAL and HCAL to determine the trigger objects, such as photons, electrons, taus and jets.
It also finds the isolated/non-isolated photons/electrons separately.
Then the global calorimeter trigger (GCT) sorts trigger objects by the energy and quality, selects top 4 of each type of the trigger objects.
It also calculates the total transverse energy and missing energy vector.
The muon local triggers are composed of three different algorithms corresponding to the different kinds of muon chambers, which are described in Sec.~\ref{ssec:muon} and Ref.~\cite{Bayatyan:706847}.
Then the Global Muon Trigger combines three local triggers and validates the muon sign, sorts muons by the energy and quality, selects top 4 of muons.
Finally, the global trigger combines all information and makes a decision.

Next, the data acquisition (DAQ) system receives the events which pass the L1 triggers and provide the computing power for the filter systems (HLT).
The structure of the DAQ is shown in Fig.~\ref{fig:CMSDAQ}.
The information will be transmitted to the DAQ, which are used, calculated by the global calorimeter trigger, the global muon trigger and the global trigger.
One zero-suppressed event occupies about 1 MByte, the data flow to the DAQ is about 100 GByte/s.
The computer farm of DAQ with thousands of commercial CPUs is designed to process such kind of huge data with more complex algorithms and performs the High Level Trigger.
The High Level Triggers is similar to the off-line reconstruction, but more efficient and lower CPU-time-consuming. 
The CMS software framework are used for the HLT and off-line reconstruction, which is open source and available in Ref.~\cite{CMSSW}.
Finally, the rate of data recorded for off-line analysis is on the order of $10^{2}$ Hz.
The more details about the HLT and DAQ are in Ref.~\cite{Cittolin:578006}

\begin{figure}[t]
  \centering
  \includegraphics[width=0.8\textwidth]{Figures/Chapter2/CMSTrigger.png}
  \caption{Architecture of the CMS L1 trigger system.}
  \label{fig:CMSTrigger}
\end{figure}

\begin{figure}[t]
  \centering
  \includegraphics[width=0.92\textwidth]{Figures/Chapter2/CMSDAQ.png}
  \caption{Architecture of the CMS DAQ system.}
  \label{fig:CMSDAQ}
\end{figure}
