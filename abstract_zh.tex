
\begin{CJK}{UTF8}{bkai}

\setstretch{1.25}
\checktoopen
\thispagestyle{plain}
\begin{center}
{\Huge \bfseries 摘 要 \par}
\end{center}
\bigskip
\bigskip
\bigskip
\bigskip
{\Large 

 本篇論文旨在尋找於質子–質子對撞下產生的雙希格斯粒子,分別衰變至一對光子及一對底夸克。
本分析使用了於2016年由大強子對撞機(LHC)產生的質子–質子對撞,總能量為$\sqrt{s}=13$ TeV,並由緊湊緲子線圈(CMS)所記錄,總亮度達到$35.9~fb^{-1}$。
此研究基於標準模型以及超越標準模型的理論,同時尋找非共振衰變的雙希格斯粒子以及由新粒子衰變的雙希格斯粒子。
非共振衰變可用於驗證希格斯機制以及探索其他可能的希格斯粒子與其他粒子的交互作用。
多維度模型預測了兩種與重力相關的新粒子,且這些新粒子可衰變到雙希格斯粒子。
本研究使用機器學習來輔助重建來自底夸克的強子噴流能量,使預測的生產截面的信心水準上限下降百分之十,達到更佳的結果。
本研究沒有觀察到顯著的訊號事件,並提供了實驗上對理論參數以及新粒子重量的限制區間。

\par}


\end{CJK}

